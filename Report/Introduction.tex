\section{Introduction}

\subsection{Problem Formulation}

The aim of this study is to investigate spatial and timing uncertainties associated with orbital rendezvous maneuvers. This is of interest because of the limited observation possibilities of orbiting objects, thus making accurate model based prediction important. In addition, the large scales in time and distance of orbital maneuvers, nonlinearities in the mechanics and the presence of disturbances make the uncertainty propagation interesting and non-trivial. \\


The theory of uncertainty propagation is a fundamental aspect of Space Situation Awareness (SSA) operations. Long-duration, high-precision uncertainty propagation is central to orbital trajectory estimation and optimization. \\


The concept of uncertainty propagation in rendezvous maneuvers is a key aspect to solving the issue of space debris, which is a major challenge of today's space industry. The number of space debris objects is expected to steadily increase over time, due to chain-reactions of collisions between existing debris objects. To prevent a runaway fragmentation of space debris, about 4 to 5 high risk objects must be actively removed from the Low Earth Orbit each year \cite{ESA_about_space_debris}. This study aims to contribute to this growing area of research by exploring the possibility of using a formation of low-cost satellites to perform Active Debris Removal. \\


\subsection{Previous Work}

In the section that follows, selected previous works on the uncertainty propagation in orbital mechanics are presented.  \\



