\section{Methods}


\subsection{Uncertainty Analysis}


\subsubsection{Error Sources}

Sources of uncertainty in space operations  will be introduced in the following section. Uncertainties in orbital estimations can originate from phenomena in the physical world, or come from approximations and deviations in the models used.  \\

Based on the review by Luo and Yang \cite{luo_review_2017}, who follow the opinion of Fehse \cite{2003ARaD}, the uncertainties related to space operation can be divided into three categories: \textit{Dynamic model errors}, \textit{actuation errors} and \textit{navigation errors}. Dynamic model errors concern the deviations in the model parameters to the real world values, such as the gravitational parameters, drag and radiation pressure. Actuation errors concern the difference between the correct actuation values and the values produced by the actuators and control system. Navigation errors are the deviations in the perceived state of the system from the actual state. An overview of the types of uncertainty in orbital mechanics is given in Tabel \ref{table:orbital_parameters}.\\



\begin{table}[h]
\centering
\begin{tabular}{@{}ll@{}}
\toprule
\textbf{Classification}                        & \textbf{Parameter}           \\ \midrule
\multirow{3}{*}{Dynamic Model Errors} & Gravitational Field \\
                                      & Drag                \\
                                      & Radiation Pressure  \\ \midrule
\multirow{3}{*}{Actuation Errors}     & Direction           \\
                                      & Timing              \\
                                      & Force               \\ \midrule
\multirow{3}{*}{Navigation Errors}    & Atmospheric Effects \\
                                      & Instrument Modeling \\
                                      & Clock Accuracy   \\    
\bottomrule
\end{tabular}%
\caption{Error Sources in Orbital Mechanics}
\label{tab:error_sources}
\end{table}
    




\subsection{Simulations}



