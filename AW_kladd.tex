\section{Assignment 2}

\textbf{Task:} \\
\textit{Think about the research area of your project. Is there a related controversy, a myth, or an issue frequently discussed in the media? (for example, renewable energy vs. fossil-based energy) } \\

\textit{Select a topic that is related to your expertise. Write an argumentative text of 450 - 500 words, presenting your informed opinion on the topic (for example: Are electric cars really so green?). Your aim is to contribute to the public debate as an expert. }  \\


* Er romsøppel noe som løser seg selv eller må vi gripe inn? \\

* Kostnad? \\

* Dagens overvåkning av situasjonen \\

* Hendelser … Ulykker og planlagte eksplosjoner \\

* Hvem sitt ansvar … ESA og NASA \\

* Livsløpsplaner for satellitter \\

* Antall romsøppel-objekter  \\

* Antall oppskytninger av satellitter \\

* Tidligere kollisjoner / hendelser \\




\textbf{Pollution Follows Humans Everywhere We Go, Even To Space} \\

\textbf{(We Need To Think Green Space)} \\

\textbf{(The Surface Of The Earth Is Covered With Debris, So Is Space)} \\


Almost 9000 satellites have been launched to space since 1957, but less than 2000 of them are operational today\cite{ESA_space_debris_figures}. \\


SpaceX plans to have a constellation of 12000 satellites in orbit by the end of 2020\cite{ESA_constellation}. \\


With the increasing number of satellites in orbit, it will at some point be practically impossible to perform "manual" collision avoidance maneuvers for all objects. \\


More than 22000 objects are regularly tracked by the American Space Surveillance Network (SSN) \cite{ESA_space_debris_figures}. \\


Researchers have found that the number of debris objects orbiting the Earth will continue to grow by itself even without any new satellites being put into orbit, through collisions between the already existing debris objects\cite{Science_2006}. \\



The risk of a collision between an operational unit and space debris increases with the number of orbiting debris objects. Because of the immense velocities of orbiting objects, even small fragments of orbiting debris pose a great risk for space missions. The velocity of objects in Low Earth Orbit (LEO) is typically more than 25 000 km/h. At this velocity, a collision with a 10 centimetre fleck of paint would have the same energy as igniting 7 kilograms of TNT\cite{kessler_syndrome_bigthink}.

The number of space debris objects is expected to steadily increase if no action is taken. Collisions between two pieces of space debris will fragment the objects; This creates a chain reaction of collisions which steadily increase the number of debris objects. This Domino effects is known as the "Kessler syndrome" after the proposal of the problem by NASA scientist Donald J. Kessler in 1978\cite{kessler_syndrome}.

To prevent a runaway fragmentation of space debris, End-of-Life disposal of new satellites put into orbit must be taken care of, and about 4 to 5 large risk objects must be actively removed from the Low Earth Orbit each year\cite{ESA_about_space_debris}.



The European Space Agency (ESA) has launched several research projects on reducing the environmental impact of space missions. ESA has developed a concept for a service satellite capable of Active Debris Removal and In-Orbit Servicing. The service satellite would be capable of rendezvousing deorbiting space debris and to perform service on operational satellites prolonging their lifespan. The cost of the project is estimated to be about 300 million Euros\cite{eDeorbit_cost}.


If no action towards reducing space debris is taken, the problem is estimated to cost the European space industry 1.5 billion Euros over the next 20 years\cite{space_debris_cost}. \\


---- \\

Examples of manouvers made to avoid collision - International Space Station (ISS) and ESA+Starlink 2019. \\

---- \\

Examples of collisions. \\

---- \\






