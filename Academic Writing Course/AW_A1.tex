\section*{Assignment 1}

\textbf{5 keywords:} orbital mechanics, nonlinearities, uncertainty propagation, satellite, rendezvous. \\

\textbf{250 word abstract:} \\
\textbf{Title: } \textit{Uncertainty model for rendezvous satellite control} \\

Uncertainties in orbit determination and the orbit uncertainties after applying rendezvous control strategies over a given time horizon for satellites has been investigated. Model assumptions, nonlinearities, and measurement uncertainties have been identified and quantified. \\

An orbit uncertainty model for the scenario of a chaser satellite rendezvousing a target object has been derived. The model is based on the parameters and thrusters of the XX satellite. Inputs to the model are: 1) The current orbit parameter observations and corresponding uncertainties of the chaser and target objects, and 2) The planned thruster firings of the chaser satellite. The model outputs the spatial and timing uncertainties of the final rendezvous between the chaser and target objects. \\

A MATLAB implementation of the proposed uncertainty model is given. Included in the report is an overview of relevant equations from orbital mechanics and statistics. \\

The derived uncertainty model provides valuable information when tuning optimal satellite controllers. The model is designed with the use case of collecting space debris in mind. The uncertainty model could be applied to a flock of satellites rendezvousing an observed piece of space debris using formation control, optimizing total energy consumption and the probability of the rendezvous under a given time frame. In this case, the uncertainty model would be an integrated part of the formation controller ensuring stability and asymptotic behavior of the system, and the uncertainty of the final satellite positions could be minimized by applying advantageous thruster firings. \\

\textbf{50 word summary:} \\
Uncertainties in orbit parameters before and after applying rendezvous control strategies for satellites has been investigated. An uncertainty propagation model for this scenario has been derived, and a Matlab implementation of the proposed algorithm is given. An overview of relevant equations from orbital mechanics and statistics is presented. \\

\textbf{One-liner:} \\
As an approach to tackling the increasing problem of space debris, I will work on designing a control algorithm utilizing a formation of small satellites to collect orbiting objects as efficiently as possible within a given time frame.
