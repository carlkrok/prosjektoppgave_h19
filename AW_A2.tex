%% This is the main file and you use this file to organize your assignment.

\documentclass[a4paper]{article}	  
\usepackage[margin=3cm]{geometry} 	   % Choose your margin here. 
\usepackage{amsmath}
\usepackage{parskip}
\usepackage{graphicx}
\usepackage{caption}
\usepackage{subcaption}
\usepackage{todonotes}
\usepackage[T1]{fontenc}
\usepackage[utf8]{inputenc}
\usepackage[english]{babel}

\newcommand{\figref}[1]{\figurename~\ref{#1}}

\let\endtitlepage\relax						% Begin the text immidiately after the title page. Optional
\setlength{\parindent}{0cm}				% Start paragraph without indent. Optional

\begin{document}

\textbf{Pollution Follows Humans Everywhere We Go, Even To Space} \\


One of the greatest challenges of today's space industry is keeping the problem of space debris under control. The evidence of space pollution can be clearly seen in the case of the more than 22,000 man-made Earth orbiting objects regularly being tracked by the American Space Surveillance Network (SSN)\cite{ESA_space_debris_figures}. The main issue with space debris is the economic and safety problems it causes. Space debris poses a risk both to human life, which is the case in manned space missions, and to the services we enjoy that are enabled by satellites, which includes communication, weather services and global navigation systems. 


Let us consider what happens if no action towards reducing space debris is taken. In the case of "business as usual", the problem of space debris is estimated to cost the European space industry 1.5 billion Euros over the next 20 years\cite{space_debris_cost}. Part of the reason for this is that even small fragments of orbiting debris pose a great risk for space missions. Because the velocity of objects in Low Earth Orbit (LEO) is typically more than 25 000 km/h, a collision between a satellite and a 10 centimetre fleck of paint would have the same energy as igniting 7 kilograms of TNT\cite{kessler_syndrome_bigthink}. In addition, the number of space debris objects is expected to steadily increase over time. Collisions between existing debris objects will create a chain reaction of debris breaking up into ever smaller fragments. This Domino effects is known as the "Kessler syndrome"\cite{kessler_syndrome}, and it poses a serious threat for the future of satellite operations.


Another significant aspect of the space debris problem is the era of satellite mega-constellations we are entering. Private companies have engaged in a modern space race, and plans have been laid for an explosive expansion in the number of satellite launches. Before looking ahead to the future of satellites in space, it is interesting to establish the current situation: Since 1957, about 9000 satellites have been launched to space, but less than 2000 of them remain operational\cite{ESA_space_debris_figures}. An example of the future of satellite launches is SpaceX's plans for a satellite constellation as part of their global internet coverage initiative \textit{Starlink}, consisting of 12,000 satellites in Low Earth Orbit within the next decade\cite{ESA_constellation}. This case demonstrates the need for better strategies for tackling the problem of space debris.\\


So far this article has focused on the need for action to constrain the problem of space debris. The following section will discuss what actions can be taken to tackle the problem, and what is currently being done. To prevent a runaway fragmentation of space debris, it is crucial that End-of-Life disposal of new satellites put into orbit is taken care of. In addition, about 4 to 5 large risk objects must be actively removed from the Low Earth Orbit each year\cite{ESA_about_space_debris}. The European Space Agency (ESA) has launched several research projects on reducing the environmental impact of space missions. ESA has developed a concept for a service satellite capable of Active Debris Removal and In-Orbit Servicing called \textit{e.Deborbit}. The service satellite would be capable of rendezvousing objects in space, deorbiting space debris and performing service on operational satellites. The cost of the project is estimated to be about 300 million Euros, with a first possible launch date in 2025\cite{eDeorbit_cost}.


\bibliographystyle{IEEEtran}
\bibliography{bibliography.bib}

\end{document}